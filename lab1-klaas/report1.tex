\documentclass[11pt,a4paper]{article}
\usepackage[utf8]{inputenc}
\usepackage{amsmath}
\usepackage{amsfonts}
\usepackage{amssymb}
\usepackage{graphicx}
\usepackage{float}
\author{Jan Kramer\\Klaas Kliffen}
\title{Computervision Lab 1}
\begin{document}
\maketitle

\section*{Exercise 1}

\section*{Exercise 2}
Matching the detail images from the street scene with the scene from David Lowe and vice-versa yields a low number of matches.
This is expected, because the detail images should not be present in the source image. The number of matches can be seen in Table \ref{tab:mismatches}.
\begin{table}[H]
\centering
\begin{tabular}{l|l|r}
Source image & detail image & Number of matches\\
\hline
scene & detail1 & 1\\
scene & detail2 & 0\\
scene & detail3 & 0\\
scene & detail4 & 2\\
scene & detail5 & 0\\
street & basmati & 1\\
street & book & 1\\
street & box & 2
\end{tabular}
\caption{Number of matches using Sift on detail images not present in the source image}
\label{tab:mismatches}
\end{table}
%TODO: Jan, kun jij hier nog een vergelijking schrijven met een goede match?
\noindent The number of matches is small with a detail image not present, compared to the number of matches on a detail image that is present. Therefor a low percentage based threshold can be used to claim a real match. 

\section*{Exercise 3}

\section*{Exercise 4}
%TODO: Jan, kun jij hier misschien nog de juiste verwijzingen naar de afbeeldingen toevoegen?
The detail book image Figure \ref{missing} is sheared and compared to the orignal scene image from Figure \ref{missing}.
The number of matches with different shear values can be seen in Table \ref{tab:sheared}.
\begin{table}[H]
\centering
\begin{tabular}{l|r}
Shear amount  & Number of matches\\
\hline
0 pixels & 98\\
25 pixels & 100\\
50 pixels & 91\\
100 pixels & 42\\
200 pixels & 3\\
300 pixels & 0\\
400 pixels & 0
\end{tabular}
\caption{Number of matches using Sift on detail sheared detail images present in the source image}
\label{tab:sheared}
\end{table}
\noindent Up to 50 pixels of shear still yields a similar amount of matches compared to the original detail image.
Higher values yield a significantly lower number of matches. Therefor it seems that Sift is not shear invariant.\\

\noindent The amount of shear which still yields tollarable results might depent on the size of the detail image. A shear of 100 pixels is almost a third of the original width of the image.

%TODO: ik weet niet of er nog meer bij kan, maar dit lijkt me het punt wat hij wil bereiken.

\end{document}